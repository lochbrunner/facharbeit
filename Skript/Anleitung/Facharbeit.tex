%%
% Einbinden der Dokumentklasse
\documentclass[12pt]{article}
%%
% Einbinden der Packages
\usepackage{german}
\usepackage{a4wide}
\usepackage{ngerman}
\usepackage[latin1]{inputenc}
\usepackage{graphicx}
\usepackage{setspace}
\usepackage{chngpage}
\usepackage{listings}
%\usepackage{float}
%\usepackage{floatflt}
\usepackage{pst-3dplot}
\usepackage{epsfig} 			%  f�r eps graphiken
\usepackage[ngerman]{babel}
\usepackage{bbm}
%%
% Einstellen 1,5fachen Zeilenabstand
\onehalfspacing

\title{Facharbeit -  Coumputer Simulation von geladenen Teilchen im elektrischen und magnetischen Feld} 
\author{Matthias Lochbrunner}
\date{\today}

%%
% Start des Dokumentes
\begin{document}

%	\pdfinfo { 
%  	/Title (Coumputer Simulation von geladenen Teilchen im elektrischen und magnetischen Feld) 
%  	/Creator (PDFLaTeX) 
%  	/Producer (SKG-Krumbach) 
%  	/Author (Matthias Lochbrunner) 
%  	/CreationDate (\today) 
%  	/ModDate (\today) 
%  	/Subject (Facharbeit 2007-2008) 
%  	/Keywords (Facharbeit) 
%	}

	%%
	% Einbinden Titelseite
	%%%%%%%%%%%%%%%%%%%%%%%%%%%%%%%%%%%%%%%%%%%%%%%%%%
%% titelseite.tex																%%
%%%%%%%%%%%%%%%%%%%%%%%%%%%%%%%%%%%%%%%%%%%%%%%%%%
%% In dieser Datei werden die Einstellungen zur %%
%% Titelseite gemacht														%%
%%%%%%%%%%%%%%%%%%%%%%%%%%%%%%%%%%%%%%%%%%%%%%%%%%

% Einstellung ohne Seitennummerierung
\thispagestyle{empty}

% Einstellen des Erstzeileneinzugs
\setlength{\parindent}{0pt}

% fetter Text, Kopf der Seite
\textbf{Simpert-Kraemer-Gymnasium \hfill 2006/2008 \\
Krumbach}

% vertikal f�llen
\vfill


\begin{center}
  \textsc{\Huge{Facharbeit}}\\  
  \large{aus dem Fach}\\ 
  \vspace*{.3cm}
  
  %%
  % Hier Fach eintragen
  \textsc{\Huge{Physik}}
  
\end{center}

% vertikal f�llen
\vfill

\begin{large}
  \textbf{Thema:}  
  %%
  % Hier Thema eintragen
  Computersimulation von geladenen Teilchen im elektrischen und magnetischen Feld
  
\end{large}
\vfill

%%
% Name, Fach, Lehrer und Datum eintragen
\begin{tabular}{ll}
  Verfasser:    & Lochbrunner Matthias\\
  Leistungskurs:& Physik \\
  Kursleiter:   & Std. Urban\\
  Abgabetermin: &  \today
\end{tabular}

% vertikal f�llen
\vfill

\begin{tabular}{lp{2cm}lp{5cm}}
	Erzielte Note: & \hrulefill & In Worten: & \hrulefill \\
			&&&\\
	Erzielte Punkte: & \hrulefill & In Worten: & \hrulefill \\
	\small{(einfache Wertung)} &&& \\
			&&&\\
			&&&\\
			&&&\\
	\multicolumn{2}{l}{Abgabe im Sekretariat:} &\multicolumn{2}{l}{\hrulefill}\\
			&&&\\
			&&&\\
	\multicolumn{2}{l}{Unterschrift des Kursleiters:} &\multicolumn{2}{l}{\hrulefill}
\end{tabular}

	%%
	% Inhaltsverzeichnis
	\tableofcontents
	
	\thispagestyle{empty}


	% Keine Nummerierung im Inhaltsverzeichnis
	\thispagestyle{empty}

	\newpage
	% Einstellen des Textlayouts auf Vorgaben f�r Facharbeiten
		\setlength{\hoffset}{-1in}
		\setlength{\voffset}{-1in}
		\setlength{\textwidth}{0cm}
		\changetext{3cm}{15cm}{2cm}{4cm}{}
		\setlength{\topmargin}{1.4cm}
	
	%%
	% Einbinden des Haupttextes
	\section{Installation}

Vor der Installation sollte sichergestellt sein, dass auf dem PC ein lauff�higes "`Windows XP"' mit Service Pack 2 installiert ist und eine aktuelle Version der Hardwareschnittstelle "`DirectX 9.0c"' oder neuer bereits Verwendung findet. Da die Simulation bereits mit dem "`DirectX SDK"' von 2007 geschrieben wurde, empfiehlt es sich gegebenfalls, die "`Runtime"' Version auf dem PC zu aktualisieren \footnote{Download : http://www.microsoft.com/downloads/details.aspx?displaylang=de\&FamilyID=2da43d38-db71-4c1b-bc6a-9b6652cd92a3}. \\
Nach Einlegen der CD in das Laufwerk erscheint automatisch ein mit "`Install Creater"' erzeugtes Installationsfenster. Sollte das Fenster nicht automatisch erscheinen, kann das Setup auf der CD auch manuell unter \textsl{/Software/Setup.exe} gestartet werden. Die Anweisungen des Installationsassistenten erkl�ren sich selbst. Nach erfolgreicher Installation erh�lt das Programm nun im Startmen�, unter \textsl{alle Progamme} einen Eintrag. \\
Ebenfalls sind auf der CD noch weitere Dateien bez�glich dieser Facharbeit zu finden. So z.B. dieses Skript als pdf-Datei, sowie der Quellcode des Programms und noch zus�tzliche Materialien.


\section{Freie Kamerabewegung}

Das Gedr�ckthalten der \textsl{Alt}-Taste signalisiert dem Programm, dass die Maus nun zur Steuerung der Kamera verwendet wird.
Jetzt l�sst sich mit der mittleren Maustaste die Szene parallel zur Sichtebene verschieben, wobei der Cursor ein anderes Symbol erh�lt, eines mit einem Pfeil in jede Himmelsrichtung. Um sich im Raum zu drehen wird die linke Maustaste verwendet. Jede Drehung erfolgt immer um einen bestimmten Mittelpunkt. Dieser ist entweder die Position des zuletzt markierten Elements in der Szene, oder falls nichts markiert wurde, der Urprung im Koordinatensystem. Als Cursor erscheinen nun zwei im Kreis laufende Pfeile.
Der Zoom bzw. der Abstand der Kamera zur Szene wird mit der rechten Maustaste ver�ndert. Hier gilt: Wird die Maus nach links oben verschoben, wird aus der Szene herausgezoomt, beim Verschieben nach rechts unten hineingezoomt. Allerdings ist zu beachten, dass der Zoom nicht linear zur Mausbewegung berechnet wird, sondern exponential. So wird ein f�r die meisten Situationen praktischeres und gleichsam harmonisches Zoomen erm�glicht.
Des �fteren kann es passieren, dass man sich irgendwo im Raum verliert und keine Orientierung mehr hat. Hierf�r ist der Hotkey \textsl{F} gedacht, welcher die Kamera auf das zuletzt markierte Element zentriert bzw. auf den Ursprung.


\section{Arbeiten im dreidimensionalen Raum}

In der Simulation stehen zwei Werkzeuge (englisch: "`Tools"') zu Verf�gung: Eines um markierte Elemente zu Verschieben und ein anderes um bestimmte Objekte, wie Kugeln und Kondensatorplatten , zu skalieren. Bei beiden Werkzeugen gilt stets die "`Dreifarbenregel"', die auch bei vielen anderen Animationsprogrammen Anwendung findet. Die drei Koordinatenachsen werden durch die drei Grundfarben repr�sentiert: Rot f�r die X-Achse, Gr�n f�r die Y-Achse und Blau f�r die Z-Achse. Mit dem Hotkey \textsl{W} wird zu dem Verschiebungswerkzeug gewechselt, mit dem Hotkey \textsl{E} zum Skalierwerkzeug. Durch erneutes Dr�cken der jeweiligen Taste wird das Werkzeug ausgeblendet. Das Skalierwerkzeug erkennt man an den drei Pfeilen, die immer in die positive Richtung zeigen. Elemente werden verschoben, indem man die Pfeilspitzen des Werkzeuges mit der linken Maustaste anklickt und die Maus dann mit gedr�ckter Taste in die gew�nschte Richtung schiebt. Oft ist es notwendig, die Szene aus dem richtigen Blickwinkel zu betrachten, um leichter arbeiten zu k�nnen. �quivalent verh�lt es sich mit dem Skalierwerkzeug, zu erkennen an den drei W�rfeln. Der Betrag der Ver�nderungen entlang der Koordinatenachsen wird mit Hilfe eines trigometrischen Ansatzes berechnet, wodurch es nach l�ngerer Zeit leider zu Unregelm��igkeiten kommen kann. Auch die wechselhafte Entfernung zur Kamera tut ihr �briges dazu.


\section{Arbeiten mit der "`Channelbox"'}

Um genaue Angaben zu einem Objekt machen zu k�nnen, wurde der Simulation ein Kontrollfenster, oder im Fachchargon auch "`Channelbox"' genannt, hinzugef�gt, mit dem der Benutzer in der Lage ist jeden Wert des markierten Elements genau angeben zu k�nnen. Das Kontrollfenster l�sst sich im \textsl{Men�} unter dem \textsl{Ansicht}, \textsl{weitere Fenster}, \textsl{Kontrollfenster} starten. Meist ist es jedoch schon zu Beginn zu sehen. Zu Beachten ist allerdings, dass nur manche Eigenschaften auf alle markierten Elemente gesetzt werden k�nnen. Angaben zu genauen Werten von z.B. Position oder Geschwindigkeit werden nur vom zuletzt markierten Element �bernommen, da dieses in der Simulation einen besonderen Fokus erh�lt.
Ebenfalls k�nnen von der \textsl{Channelbox} stets die genauen Werte abgelesen werden, die in einer �hnlichen Form, wie in den *.sim-Datein dargestellt sind. Dateien mit der Endung .sim werden dazu benutzt ein Szene abzuspeichern, welche zu einem sp�teren Zeitpunkt dann wieder geladen werden kann. Aufgrund technischer Schwierigkeiten, ist es bereits noch nicht gelungen diese Dateien direkt mit dem Programm �ffnen zu lassen. Aber dennoch eignet sich dieses Dateiformat ausgezeichnet, es mit dem Microsoft Text-Editor oder irgendeinem anderen Text-Editor zu �ffnen, um damit den Inhalt zu ver�ndern. So lasses sich beispielswei�e ohne das Starten der Anwendung, neue Elemente zu erstellen ud sie mit den gew�nschten Werten zu versehen.


\section{Szene animieren}

Dem eigentlichen Zweck als Simulation gen�ge zu tun und der 3D-Szene Leben einzuhauchen, kann mithilfe der Funktionstaste \textsl{F2} die Animation gestartet werden, in der dann automatisch alle physikalischen Berechungen durchgef�hrt werden. Durch erneutes Dr�cken der der Funktionstaste wird die Animation wieder gestoppt. Mit der \textsl{TAB}-Taste wird erreicht, dass nur das n�chste Bild, auf englisch : "`Frame"', berechnet und angezeigt wird. So l�sst sich die Animation schrittwei�e abspielen, wobei der Benutzer in komplexen Szenen meist besser in der Lage ist, den �berblick zu behalten.\\
Eine weitere Erleichterung ist mit der Funktion, die Bahn eines Teilchens anzeigen zu lassen, gegeben. Somit l�sst sich das Verhalten des Spurauslegers genauer analysieren und der Bertachter bekommt ein aussagekr�ftiges Bild von der Szene.

	
	%%
% Tabellen- und Abbildungsverzeichnis.
% Falls ben�tigt, einfach die %-Zeichen vor
% den Befehlen entfernen
%%
		%\listoftables
		%\listoffigures
		%\lstlistoflistings

	%%
	% Einbinden der Schlusserkl�rung
	%
Ich erkl�re hiermit, dass ich die Facharbeit ohne fremde Hilfe angefertigt und nur die im Literaturverzeichnis angef�hrten Quellen und Hilfsmittel benutzt habe. \\ \vspace{2cm}

\begin{tabular}{p{5.5cm}p{3cm}lp{5cm}l}
\hrulefill , den		& \hrulefill 		&& \hrulefill\\
Ort 								& 	 Datum 			&& Unterschrift
\end{tabular}

\end{document}
