%%
% Einbinden der Dokumentklasse
\documentclass[12pt]{article}
%%
% Einbinden der Packages
\usepackage{german}
\usepackage{a4wide}
\usepackage{ngerman}
\usepackage[latin1]{inputenc}
\usepackage{graphicx}
\usepackage{setspace}
\usepackage{chngpage}
\usepackage{listings}
%\usepackage{float}
%\usepackage{floatflt}
\usepackage{pst-3dplot}
\usepackage{epsfig} 			%  f�r eps graphiken
\usepackage[ngerman]{babel}
\usepackage{bbm}
%\usepackege{amssymb}
%%
% Einstellen 1,5fachen Zeilenabstand
\onehalfspacing

\title{Facharbeit -  Coumputer Simulation von geladenen Teilchen im elektrischen und magnetischen Feld} 
\author{Matthias Lochbrunner}
\date{\today}

%%
% Start des Dokumentes
\begin{document}

%	\pdfinfo { 
%  	/Title (Coumputer Simulation von geladenen Teilchen im elektrischen und magnetischen Feld) 
%  	/Creator (PDFLaTeX) 
%  	/Producer (SKG-Krumbach) 
%  	/Author (Matthias Lochbrunner) 
%  	/CreationDate (\today) 
%  	/ModDate (\today) 
%  	/Subject (Facharbeit 2007-2008) 
%  	/Keywords (Facharbeit) 
%	}

	%%
	% Einbinden Titelseite
	%%%%%%%%%%%%%%%%%%%%%%%%%%%%%%%%%%%%%%%%%%%%%%%%%%
%% titelseite.tex																%%
%%%%%%%%%%%%%%%%%%%%%%%%%%%%%%%%%%%%%%%%%%%%%%%%%%
%% In dieser Datei werden die Einstellungen zur %%
%% Titelseite gemacht														%%
%%%%%%%%%%%%%%%%%%%%%%%%%%%%%%%%%%%%%%%%%%%%%%%%%%

% Einstellung ohne Seitennummerierung
\thispagestyle{empty}

% Einstellen des Erstzeileneinzugs
\setlength{\parindent}{0pt}

% fetter Text, Kopf der Seite
\textbf{Simpert-Kraemer-Gymnasium \hfill 2006/2008 \\
Krumbach}

% vertikal f�llen
\vfill


\begin{center}
  \textsc{\Huge{Facharbeit}}\\  
  \large{aus dem Fach}\\ 
  \vspace*{.3cm}
  
  %%
  % Hier Fach eintragen
  \textsc{\Huge{Physik}}
  
\end{center}

% vertikal f�llen
\vfill

\begin{large}
  \textbf{Thema:}  
  %%
  % Hier Thema eintragen
  Computersimulation von geladenen Teilchen im elektrischen und magnetischen Feld
  
\end{large}
\vfill

%%
% Name, Fach, Lehrer und Datum eintragen
\begin{tabular}{ll}
  Verfasser:    & Lochbrunner Matthias\\
  Leistungskurs:& Physik \\
  Kursleiter:   & Std. Urban\\
  Abgabetermin: &  \today
\end{tabular}

% vertikal f�llen
\vfill

\begin{tabular}{lp{2cm}lp{5cm}}
	Erzielte Note: & \hrulefill & In Worten: & \hrulefill \\
			&&&\\
	Erzielte Punkte: & \hrulefill & In Worten: & \hrulefill \\
	\small{(einfache Wertung)} &&& \\
			&&&\\
			&&&\\
			&&&\\
	\multicolumn{2}{l}{Abgabe im Sekretariat:} &\multicolumn{2}{l}{\hrulefill}\\
			&&&\\
			&&&\\
	\multicolumn{2}{l}{Unterschrift des Kursleiters:} &\multicolumn{2}{l}{\hrulefill}
\end{tabular}

	%%
	% Inhaltsverzeichnis
	\tableofcontents
	
	\thispagestyle{empty}


	% Keine Nummerierung im Inhaltsverzeichnis
	\thispagestyle{empty}

	\newpage
	% Einstellen des Textlayouts auf Vorgaben f�r Facharbeiten
		\setlength{\hoffset}{-1in}
		\setlength{\voffset}{-1in}
		\setlength{\textwidth}{0cm}
		\changetext{3cm}{15cm}{2cm}{4cm}{}
		\setlength{\topmargin}{1.4cm}
	
	%%
	% Einbinden des Haupttextes
	\section {Einf�hrung zum Thema}
 
Zu Zeiten Isaac Newtons war die Physik noch ein sehr �berschaubares Gebiet der Wissenschaft. Als Rechenhandwerk gen�gten primitivste Differenzial- und Integralrechnung vollkommen. Gleichungen dritten Grades waren eine Seltenheit. Und war mal ein neuartiges Ph�nomen gefunden, dass mit den bekannten Gesetzen nicht beschrieben werden konnte, bot sich immer noch die M�glichkeit die fehlende Erkenntisse durch ein Experiment zu erlangen. Der Einfallsreichtum unter den Versuchsaufbauten war oftmals erstaunlich.
Doch sp�testens seit dem ersten Versuch, einen Menschen auf den Mond zu schie�en, w�re es ein recht teueres Unterfangen gewesen eine Vielzahl von Raketen gen Himmel zu schicken nur um herauszufinden, wie stark die Triebwerksk�pfe sein m�ssten um das Vehikel zur gew�nschten Umlaufbahn zu bef�rdern. Hier wurde dann auch erstmals das Berechnen von Koordinaten und Kreisbahnen mit Papier und Bleistift ein Ding der Unm�glichkeit. Nicht nur weil die Rechnungen eines einzelnen Teilabschnittes unerm��lich lang und  aufw�ndig wurde, sondern auch weil niemand die Verantwortung �bernehmen wollte, wegen einem Rechenfehler ein Menschenleben zu riskieren.
Ein neuer Weg musste gefunden werden, solch komplexe Berechungen schnell und exakt durchf�hren zu k�nnen. Man mag die Wissenschaftler wohl noch ausgelacht haben, als sie mit Lochstreifen bewaffnet den ersten R�hrenrie�en das multiplizieren beibrachten um ihnen nach monatelangem gut Zureden dazu zu bringen, die Kurven m�glicher Flugbahnen auszuspucken.
Diese Hilfe verwendet man heute auch "`in der Elementarteilchenphysik, in der Materialforschung, Str�mungsdynamik, Strukturmechanik, Chemie, Geo- und Astrophysik sowie Klima- und Umweltforschung"'\footnote{Quelle : http://www.stmwfk.bayern.de/pressearchiv/meldung.asp?NewsID=649}.
Dazu reichen heutzutage allerdings keine Triodenrechner mehr, die in einer Sekunde eine Handvoll Rechenschritte durchf�hren, sondern es werden meist Gro�rechner wie der Leibnizrechner in Garching ben�tigt, um die rie�ige Datenflut und die �beraus komplizierten Berechnungen, die dennoch nur Ann�herungen sind, zu bew�ltigen.\\
Aber auch im Kleinen lassen sich durch Computersimulationen viele physikalische Ph�nomen anschaulich erkl�ren. Eine gro�e Hilfe f�r Physiklehrer, die im Unterricht bei Sch�lern oft auf Unverst�ndis und mangelnder Vorstellungskraft treffen.
Die Vorteile des Computereinsatzes im Unterricht liegen klar auf der Hand: So lassen sich durch eine geeignete Simulation das komplexe Zusammenspiel vieler Umwelteinfl�sse auf einzellne wichtige Faktoren begrenzen und diese komplizierten Sachverhalte durch transpatente Modelle durchleuchten, da unn�tige Einfl�sse ausgeblendet werden. Idealbedingungen k�nnen geschaffen werden. Auch erm�glichen diese das beobachten physiklaischer Experimente die real im Unterrricht niemals durchf�hrbar w�ren, wie zum Beispiel sehr langsame bzw. sehr schnell ablaufende Effekte wie ein elektischer Blitz, ebenso Versuche mit strahlenden oder hoch giftigen Materialien w�ren im Klassenzimmer unverantwortlich mit einem Computersimualtion jedoch anschaulich durchf�hrbar\footnote{Quelle : http://www.medien.ifi.lmu.de/fileadmin/mimuc/mll\_ws0506/Vortrag\_Kim.pdf}.
Zwar bietet das Internet ein reichhaltiges Angebot physikalischer Simualtionsprogramme, oft sogar als Java-Applet oder Flash-Skript direkt im Internetbrowser ausf�hrbar, doch nur die Wenigsten von ihnen arbeiten mit der dritten Dimension, und alle haben nur einen beschr�nkten Anwendungsbereich. Da sie meist nur geskriptet sind, bekommt der Betrachter quasi nur einen Film vorgesetzt, bei dem er manche Werte mit Reglern verstellen kann. Diese Nische zu f�llen entschied ich mich, selber im Rahmen dieser Facharbeit, ein Programm zu entwickeln, das auf der Basis von Microsoft DirectX\texttrademark geschrieben in \textsl{C++}, phyikalische Ph�nomene in den Bereichen Elektrostatik und Magnetismus im dreidimensionalem Raum simulieren kann.
Hierf�r setze ich auf das bewerte Editor-Prinzip, welches dem Anwender erlaubt sich aus einer Reihe von Bausteinen seine gew�nschte Szene zusammen zu stellen, genauso wie es in komerziellen 3D-Simulationen wie Maya oder Cinema3D �blich ist.
Somit lassen sich dann die Verhaltensmuster Elementarteichen in bestimmten Umgebungen beobachten; genauere Ph�nomene werden am Schluss dieser Arbeit behandelt. Zun�chst werde ich kurz die Handhabung und den Umfang des Programms vorstellen, danach noch auf die physikalischen Gesetze eingehen, die in dieser Simulation Anwendung finden und wie sie im Code umgesetzt wurden.
Die Angabe genauer Ergebisse und realit�tsgetreuer Werte, die genau berechnete Konstanten im Quellcode vorraussetzt, ist nicht Aufgabe dieser Facharbeit. Und somit handelt es sich um eine reine Simulation ohne Messm�glichkeit.\\

	\section{Installation}

Vor der Installation sollte sichergestellt sein, dass auf dem PC ein lauff�higes "`Windows XP"' mit Service Pack 2 installiert ist und eine aktuelle Version der Hardwareschnittstelle "`DirectX 9.0c"' oder neuer bereits Verwendung findet. Da die Simulation bereits mit dem "`DirectX SDK"' von 2007 geschrieben wurde, empfiehlt es sich gegebenfalls, die "`Runtime"' Version auf dem PC zu aktualisieren \footnote{Download : http://www.microsoft.com/downloads/details.aspx?displaylang=de\&FamilyID=2da43d38-db71-4c1b-bc6a-9b6652cd92a3}. \\
Nach Einlegen der CD in das Laufwerk erscheint automatisch ein mit "`Install Creater"' erzeugtes Installationsfenster. Sollte das Fenster nicht automatisch erscheinen, kann das Setup auf der CD auch manuell unter \textsl{/Software/Setup.exe} gestartet werden. Die Anweisungen des Installationsassistenten erkl�ren sich selbst. Nach erfolgreicher Installation erh�lt das Programm nun im Startmen�, unter \textsl{alle Progamme} einen Eintrag. \\
Ebenfalls sind auf der CD noch weitere Dateien bez�glich dieser Facharbeit zu finden. So z.B. dieses Skript als pdf-Datei, sowie der Quellcode des Programms und noch zus�tzliche Materialien.


\section{Freie Kamerabewegung}

Das Gedr�ckthalten der \textsl{Alt}-Taste signalisiert dem Programm, dass die Maus nun zur Steuerung der Kamera verwendet wird.
Jetzt l�sst sich mit der mittleren Maustaste die Szene parallel zur Sichtebene verschieben, wobei der Cursor ein anderes Symbol erh�lt, eines mit einem Pfeil in jede Himmelsrichtung. Um sich im Raum zu drehen wird die linke Maustaste verwendet. Jede Drehung erfolgt immer um einen bestimmten Mittelpunkt. Dieser ist entweder die Position des zuletzt markierten Elements in der Szene, oder falls nichts markiert wurde, der Urprung im Koordinatensystem. Als Cursor erscheinen nun zwei im Kreis laufende Pfeile.
Der Zoom bzw. der Abstand der Kamera zur Szene wird mit der rechten Maustaste ver�ndert. Hier gilt: Wird die Maus nach links oben verschoben, wird aus der Szene herausgezoomt, beim Verschieben nach rechts unten hineingezoomt. Allerdings ist zu beachten, dass der Zoom nicht linear zur Mausbewegung berechnet wird, sondern exponential. So wird ein f�r die meisten Situationen praktischeres und gleichsam harmonisches Zoomen erm�glicht.
Des �fteren kann es passieren, dass man sich irgendwo im Raum verliert und keine Orientierung mehr hat. Hierf�r ist der Hotkey \textsl{F} gedacht, welcher die Kamera auf das zuletzt markierte Element zentriert bzw. auf den Ursprung.


\section{Arbeiten im dreidimensionalen Raum}

In der Simulation stehen zwei Werkzeuge (englisch: "`Tools"') zu Verf�gung: Eines um markierte Elemente zu Verschieben und ein anderes um bestimmte Objekte, wie Kugeln und Kondensatorplatten , zu skalieren. Bei beiden Werkzeugen gilt stets die "`Dreifarbenregel"', die auch bei vielen anderen Animationsprogrammen Anwendung findet. Die drei Koordinatenachsen werden durch die drei Grundfarben repr�sentiert: Rot f�r die X-Achse, Gr�n f�r die Y-Achse und Blau f�r die Z-Achse. Mit dem Hotkey \textsl{W} wird zu dem Verschiebungswerkzeug gewechselt, mit dem Hotkey \textsl{E} zum Skalierwerkzeug. Durch erneutes Dr�cken der jeweiligen Taste wird das Werkzeug ausgeblendet. Das Skalierwerkzeug erkennt man an den drei Pfeilen, die immer in die positive Richtung zeigen. Elemente werden verschoben, indem man die Pfeilspitzen des Werkzeuges mit der linken Maustaste anklickt und die Maus dann mit gedr�ckter Taste in die gew�nschte Richtung schiebt. Oft ist es notwendig, die Szene aus dem richtigen Blickwinkel zu betrachten, um leichter arbeiten zu k�nnen. �quivalent verh�lt es sich mit dem Skalierwerkzeug, zu erkennen an den drei W�rfeln. Der Betrag der Ver�nderungen entlang der Koordinatenachsen wird mit Hilfe eines trigometrischen Ansatzes berechnet, wodurch es nach l�ngerer Zeit leider zu Unregelm��igkeiten kommen kann. Auch die wechselhafte Entfernung zur Kamera tut ihr �briges dazu.


\section{Arbeiten mit der "`Channelbox"'}

Um genaue Angaben zu einem Objekt machen zu k�nnen, wurde der Simulation ein Kontrollfenster, oder im Fachchargon auch "`Channelbox"' genannt, hinzugef�gt, mit dem der Benutzer in der Lage ist jeden Wert des markierten Elements genau angeben zu k�nnen. Das Kontrollfenster l�sst sich im \textsl{Men�} unter dem \textsl{Ansicht}, \textsl{weitere Fenster}, \textsl{Kontrollfenster} starten. Meist ist es jedoch schon zu Beginn zu sehen. Zu Beachten ist allerdings, dass nur manche Eigenschaften auf alle markierten Elemente gesetzt werden k�nnen. Angaben zu genauen Werten von z.B. Position oder Geschwindigkeit werden nur vom zuletzt markierten Element �bernommen, da dieses in der Simulation einen besonderen Fokus erh�lt.
Ebenfalls k�nnen von der \textsl{Channelbox} stets die genauen Werte abgelesen werden, die in einer �hnlichen Form, wie in den *.sim-Datein dargestellt sind. Dateien mit der Endung .sim werden dazu benutzt ein Szene abzuspeichern, welche zu einem sp�teren Zeitpunkt dann wieder geladen werden kann. Aufgrund technischer Schwierigkeiten, ist es bereits noch nicht gelungen diese Dateien direkt mit dem Programm �ffnen zu lassen. Aber dennoch eignet sich dieses Dateiformat ausgezeichnet, es mit dem Microsoft Text-Editor oder irgendeinem anderen Text-Editor zu �ffnen, um damit den Inhalt zu ver�ndern. So lasses sich beispielswei�e ohne das Starten der Anwendung, neue Elemente zu erstellen ud sie mit den gew�nschten Werten zu versehen.


\section{Szene animieren}

Dem eigentlichen Zweck als Simulation gen�ge zu tun und der 3D-Szene Leben einzuhauchen, kann mithilfe der Funktionstaste \textsl{F2} die Animation gestartet werden, in der dann automatisch alle physikalischen Berechungen durchgef�hrt werden. Durch erneutes Dr�cken der der Funktionstaste wird die Animation wieder gestoppt. Mit der \textsl{TAB}-Taste wird erreicht, dass nur das n�chste Bild, auf englisch : "`Frame"', berechnet und angezeigt wird. So l�sst sich die Animation schrittwei�e abspielen, wobei der Benutzer in komplexen Szenen meist besser in der Lage ist, den �berblick zu behalten.\\
Eine weitere Erleichterung ist mit der Funktion, die Bahn eines Teilchens anzeigen zu lassen, gegeben. Somit l�sst sich das Verhalten des Spurauslegers genauer analysieren und der Bertachter bekommt ein aussagekr�ftiges Bild von der Szene.

	
	%%
% Tabellen- und Abbildungsverzeichnis.
% Falls ben�tigt, einfach die %-Zeichen vor
% den Befehlen entfernen
%%
		%\listoftables
		\listoffigures
		\lstlistoflistings
		
		\vspace{1 cm}

\begin{Large}\textbf{Verwendete Quellen}\end{Large}

\begin{enumerate}
	\item  \hspace{0.5 cm}http://www.stmwfk.bayern.de/pressearchiv/meldung.asp?NewsID=649 \hfill 3
	\item  \hspace{0.5 cm}http://www.medien.ifi.lmu.de/fileadmin/mimuc/mll ws0506/Vortrag Kim.pdf \hfill 4
	\item  \hspace{0.5 cm}Physik, Leistugskurs 1. Semester, Seite 61  \hfill 8
	\item  \hspace{0.5 cm}Physik, Leistugskurs 1. Semester, Seite 110/111 \hfill 9
	\item  \hspace{0.5 cm}http://www-history.mcs.st-andrews.ac.uk/Biographies/Heun.html \hfill 14
	\item  \hspace{0.5 cm}http://www.acdca.ac.at/material/kl8/numerik.pdf \hfill 15
\end{enumerate}

	%%
	% Einbinden der Schlusserkl�rung
	
Ich erkl�re hiermit, dass ich die Facharbeit ohne fremde Hilfe angefertigt und nur die im Literaturverzeichnis angef�hrten Quellen und Hilfsmittel benutzt habe. \\ \vspace{2cm}

\begin{tabular}{p{5.5cm}p{3cm}lp{5cm}l}
\hrulefill , den		& \hrulefill 		&& \hrulefill\\
Ort 								& 	 Datum 			&& Unterschrift
\end{tabular}

\end{document}
